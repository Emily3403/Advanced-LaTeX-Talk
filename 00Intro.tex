%&template
\documentclass[main.tex]{subfiles}

\begin{document}

\begin{frame}
    \titlepage
\end{frame}

\begin{frame}{Gliederung des Talks}
    \begin{tcolorbox}[colback=red!7,colframe=red!50!black,rounded corners]
        \vspace{5pt}
        \tableofcontents[sections={1-3}]
    \end{tcolorbox}
    \vspace{10pt}
    \begin{tcolorbox}[colback=red!7,colframe=red!50!black,rounded corners]
        \vspace{5pt}
        \tableofcontents[sections={4-6}]
    \end{tcolorbox}
\end{frame}

\begin{frame}{Intro}
    \begin{itemize}
        \item \LaTeX{} ist wie das Javascript unter den Markup-Sprachen
        \pause
        \begin{itemize}
            \item Historisch gewachsen
            \pfeil{Kann sehr frustrierend sein}
            \pause
            \item Sehr Flexibel
            \pause
            \item Es gibt immer eine Lösung\pause, auch, wenn sie nicht schön ist
        \end{itemize}
        \pause
        \medskip
        \item \LaTeX{} wird La-Tech ausgesprochen, nicht Latex
        \pause
        \item \LaTeX{} ist sehr viel learning-by-doing
        \pause
        \medskip
        \item Google \only<9-10>{(oder ChatGPT) }ist der beste Freund  \phantom{(}  % We need the phantom to provide the vertical space. Otherwise the animation will not be smooth
        \pause
        \pause
        \item Diese Präsentation (+ mehr Materialen) werden hochgeladen
        \pfeil{\href{https://wiki.freitagsrunde.org/TechTalks}{Freitagsrunde}, \href{https://github.com/Emily3403/Advanced-LaTeX-Talk}{GitHub}}
    \end{itemize}
\end{frame}


\end{document}

