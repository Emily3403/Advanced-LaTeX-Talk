\documentclass[main.tex]{subfiles}

\begin{document}

\section{Schnelle Compiles mit \LaTeX}

\begin{frame}{Schnelle Compiles: Warum?}
    \only<1-4, 10-14>{
        \begin{itemize}
        \only<1-5, 10->{\item \LaTeX{} wird kompiliert, es ist nicht WYSIWYG}
        \onslide<2-5, 10->{\item Bis das Dokument kompiliert ist, gibt es kein Feedback}
        \onslide<3-5, 10->{
            \begin{itemize}
            \item Positionierung von Bildern
            \item TikZ Grafiken
            \item \ldots
            \end{itemize}
        }
        \medskip
        \onslide<4-, 10->{\item Feedback Cycle}
        \medskip
        \onslide<11->{\item Verschiedene Optimierungsdimensionen}
        \begin{enumerate}
        \onslide<12->{\item Beschleunigen des Codes}
        \onslide<13->{\item Reduzieren der zu verarbeitenden Code-Menge}
        \onslide<14->{\item TikZ}
        \end{enumerate}
        \end{itemize}
    }

    \only<5-9>{
        \begin{center}
            \begin{tikzpicture}[
                node distance = 3cm,
                auto,
                every node/.style={align=center, font=\small},
                process/.style={draw, circle, rounded corners, inner sep=0pt, minimum width=42pt},
                arrowlabel/.style={midway, align=center, font=\footnotesize}
            ]

                % Nodes
                \node[process, initial by arrow, initial text=] (edit) {Edit};
                \node[process, right=of edit] (compile) {Compile};
                \node[process, below=of $(edit)!0.5!(compile)$] (review) {Review};

                % Arrows
                \onslide<6-9>{\draw[->, thick, bend left=45] (edit) to node[arrowlabel] {Start compilation} (compile);}

                \onslide<7-9>{\draw[->, thick, bend left=45] (compile) to node[arrowlabel] {Wait for\\compilation} (review);}
                \onslide<9>{\draw[->, thick, red, bend left=45] (compile) to node[arrowlabel] {Wait for\\compilation} (review);}

                \onslide<8-9>{\draw[->, thick, bend left=45] (review) to node[arrowlabel, pos=0.4] {Respond to\\changes} (edit);}

            \end{tikzpicture}
        \end{center}
    }

\end{frame}

\begin{frame}{Schnelle Compiles: Beschleunigen des Codes}
    \begin{itemize}
        \item Vermeiden von unnötigen Paketen
        \begin{itemize}
            \item TikZ
            \item Bib(La)\TeX
            \item minted
        \end{itemize}
        \pause
        \medskip
        \item \href{https://tex.stackexchange.com/a/8793}{Trojanische Pferde} vermeiden
        \begin{itemize}
            \item Pakete, die als Dependency ein langsames Paket haben
        \end{itemize}
        \pause
        \medskip
        \item Caches nutzen: Build Artefakte nicht löschen
    \end{itemize}
\end{frame}

\begin{frame}[fragile]{Schnelle Compiles: Optimieren von Grafiken}
    \begin{itemize}
        \item Auflösung anpassen
        \begin{itemize}
            \item Reduzieren der Auflösung (4k $\to$ 1080p)
            \item[$\to$] Schnellere Verarbeitung
        \end{itemize}
        \pause
        \medskip
        \item In \texttt{.pdf} konvertieren
        \begin{itemize}
            \item Natives Format für \LaTeX-Ausgabe
            \item Keine zusätzliche Formatkonvertierung
            \item[$\to$] Kürzere Kompilierungszeit
        \end{itemize}
        \pause
        \medskip
        \item Kann automatisch gemacht werden:
        \begin{Verbatim}
find . -type f -exec file --mime-type {} \+  |
awk -F: '{if ($2 ~/image\//) print $1}'      |
xargs -P "$(nproc)" -I it                    \
sh -c 'img2pdf $1 -o ${1%.*}.pdf' -- it $argv
        \end{Verbatim}
    \end{itemize}
\end{frame}

\begin{frame}[fragile]{Schnelle Compiles: Reduzieren der Code-Menge}
    \begin{itemize}
        \item Beispiel: Dokument von 100 Seiten
        \item Meist wird nur 1 aktiv bearbeitet, warum alle kompilieren?
        \pause
        \bigskip
        \item \verb|\usepackage{subfiles}|: Zerteilen des Projekts
        \item Jedes Teildokument kann separat kompiliert werden
        \pause
        \bigskip
        \item Jedes Teildokument ist ein vollständiges \LaTeX-Dokument
        \item Die main Datei bekommt den Inhalt der Subfile
        \item Die Subfile bekommt die Präambel der main Datei
        \pause
        \bigskip
        \item Wird transparent von \verb|\documentclass{subfiles}| gemacht
        \item Minimaler Overhead
    \end{itemize}
\end{frame}


\begin{frame}[fragile]{Schnelle Compiles: Precompiled Header}
    \begin{itemize}
        \item Die Präambel des Dokuments ändert sich (fast) nie
        \item Warum wird die immer wieder aufs neue kompiliert?
        \pause
        \medskip
        \item Abspeichern des Zustands am Ende der Präambel
        \pfeil{Kann bei der nächsten Kompilation geladen werden}
        \pause
        \medskip
        \item Mit dem Paket \texttt{mylatexformat} möglich:
        \begin{verbatim}
pdflatex -ini -jobname="main" "&pdflatex" \
mylatexformat.ltx main.tex
        \end{verbatim}
        \pause
        \vspace{-15pt}
        \item Heraus kommt eine $\sim$10MB Binärdatei
        \item Kann über den Header \verb|%&main| eingebunden werden
    \end{itemize}
\end{frame}

\begin{frame}{Schnelle Compiles: Externalisierung von TikZ Bildern}
    \begin{itemize}
        \item Beispiel: Dokument mit 20 schönen TikZ Bildern.
        \item Keins ist unnötig, alle sollen gerendert werden
        \pause
        \pfeil{Externalisierung nutzen}
        \medskip
        \pause
        \item TikZ-Bilder in separate PDFs speichern
        \pause
        \item Einfügen der PDFs in Hauptdokument
        \pause
        \item Beschleunigung durch Vermeidung von Neuberechnungen
        \pause
        \item Aktualisierung nur bei Änderungen erforderlich
        \pause
        \medskip
        \item Mit \texttt{[list and make]} ist parallele Verarbeitung möglich!
        \pfeil{Bisschen komplizierter, \href{https://tikz.dev/library-external}{zum lesen}}
    \end{itemize}
\end{frame}

\begin{frame}[fragile]{Schnelle Compiles: Fallbeispiel}
    \begin{itemize}
        \item Diese Präsentation ist mit \LaTeX-Beamer erstellt
        \pause
        \item Es dauert 32s um die Präsentation \textit{vollständig} zu kompilieren
        \pause
        \pfeil{3 compile Runs + 1 biber Run}
        \medskip
        \pause
        \item Ein compile Run dauert 7s
        \pause
        \item Mit Format-File (Template) 6s 500ms
    \end{itemize}
    \pause
    \begin{tabularx}{\textwidth}{C{50pt}*{2}X}
        \toprule
        Subfile & Ohne Template & Mit Template \\
        \midrule
        0       & 1s 300ms      & 600ms        \\
        1       & 3s            & 2s 200ms     \\
        2       & 1s 400ms      & 730ms        \\
        3       & 2s            & 1s 200ms     \\
        4       & 1s 400ms      & 700ms        \\
        5       & 1s 300ms      & 600ms        \\
        6       & 3s 300ms      & 2s 500ms     \\
        \bottomrule
    \end{tabularx}
\end{frame}


\end{document}
